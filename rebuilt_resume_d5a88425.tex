\documentclass[letterpaper,11pt]{article}

\usepackage{latexsym}
\usepackage[empty]{fullpage}
\usepackage{titlesec}
\usepackage{marvosym}
\usepackage[usenames,dvipsnames]{color}
\usepackage{verbatim}
\usepackage{enumitem}
\usepackage[hidelinks]{hyperref}
\usepackage{fancyhdr}
\usepackage[english]{babel}
\usepackage{tabularx}
\usepackage{fontawesome5}
\usepackage{multicol}
\setlength{\multicolsep}{-3.0pt}
\setlength{\columnsep}{-1pt}
\input{glyphtounicode}

\pagestyle{fancy}
\fancyhf{} 
\fancyfoot{}
\renewcommand{\headrulewidth}{0pt}
\renewcommand{\footrulewidth}{0pt}

% Adjust margins
\addtolength{\oddsidemargin}{-0.6in}
\addtolength{\evensidemargin}{-0.5in}
\addtolength{\textwidth}{1.19in}
\addtolength{\topmargin}{-.7in}
\addtolength{\textheight}{1.4in}

\urlstyle{same}

\raggedbottom
\raggedright
\setlength{\tabcolsep}{0in}

% Sections formatting
\titleformat{\section}{
  \vspace{-4pt}\scshape\raggedright\large\bfseries
}{}{0em}{}[\color{black}\titlerule \vspace{-5pt}]

% Ensure that generate pdf is machine readable/ATS parsable
\pdfgentounicode=1

%-------------------------
% Custom commands
\newcommand{\resumeItem}[1]{
  \item\small{
    {#1 \vspace{-2pt}}
  }
}

\newcommand{\classesList}[4]{
    \item\small{
        {#1 #2 #3 #4 \vspace{-2pt}}
  }
}

\newcommand{\resumeSubheading}[4]{
  \vspace{-2pt}\item
    \begin{tabular*}{1.0\textwidth}[t]{l@{\extracolsep{\fill}}r}
      \textbf{#1} & \textbf{\small #2} \\
      \textit{\small#3} & \textit{\small #4} \\
    \end{tabular*}\vspace{-7pt}
}

\newcommand{\resumeSubSubheading}[2]{
    \item
    \begin{tabular*}{0.97\textwidth}{l@{\extracolsep{\fill}}r}
      \textit{\small#1} & \textit{\small #2} \\
    \end{tabular*}\vspace{-7pt}
}

\newcommand{\resumeProjectHeading}[2]{
    \item
    \begin{tabular*}{1.001\textwidth}{l@{\extracolsep{\fill}}r}
      \small#1 & \textbf{\small #2}\\
    \end{tabular*}\vspace{-7pt}
}

\newcommand{\resumeSubItem}[1]{\resumeItem{#1}\vspace{-4pt}}

\renewcommand\labelitemi{$\vcenter{\hbox{\tiny$\bullet$}}$}
\renewcommand\labelitemii{$\vcenter{\hbox{\tiny$\bullet$}}$}

\newcommand{\resumeSubHeadingListStart}{\begin{itemize}[leftmargin=0.0in, label={}]}
\newcommand{\resumeSubHeadingListEnd}{\end{itemize}}
\newcommand{\resumeItemListStart}{\begin{itemize}}
\newcommand{\resumeItemListEnd}{\end{itemize}\vspace{-5pt}}
\newcommand\sbullet[1][.5]{\mathbin{\vcenter{\hbox{\scalebox{#1}{\tiny$\bullet$}}}}}
\newcommand{\descript}[1]{\color{subheadings}\raggedright\hspace*{0pt}\hfill\vspace{3pt}\fontsize{11pt}{13pt}\selectfont {#1 \\} \normalfont}

\begin{document}

\begin{center}
    {\Huge \scshape Liam Johnson} \\ \vspace{1pt}
    \large{Aspiring Ceramic Engineer} \\ \vspace{4pt}
    \small \raisebox{-0.1\height}\faMapMarkerAlt\ San Diego, California ~ \raisebox{-0.1\height}\faPhone\ +1 (234) 555-1234 ~ \href{mailto:help@enhancv.com}{\raisebox{-0.2\height}\faEnvelope\  \underline{help@enhancv.com}} ~ 
    \href{https://linkedin.com/in/liam-johnson}{\raisebox{-0.2\height}\faLinkedin\ \underline{linkedin.com/in/liam-johnson}}
    \vspace{-8pt}
\end{center}

%-----------EDUCATION-----------
\section{Education}
  \resumeSubHeadingListStart
    \resumeSubheading
      {University of California San Diego}{January 2019 -- January 2023}
      {Bachelor of Science in Materials Science and Engineering}{San Diego, California}
  \resumeSubHeadingListEnd

%------RELEVANT COURSEWORK-------
\section{Relevant Coursework}
    \begin{multicols}{3}
        \begin{itemize}[itemsep=-5pt, parsep=3pt]
            \item\small Ceramic Processing and Design
            \item\small Thermodynamics of Materials
            \item\small Materials Characterization
            \item\small Mechanical Behavior of Materials
            \item\small Kiln and Furnace Technology
            \item\small Phase Transformations
        \end{itemize}
    \end{multicols}
    \vspace*{2.0\multicolsep}
    
%-----------Experience-----------%
\section{Work Experience}
  \resumeSubHeadingListStart
    \resumeSubheading
      {Materials Science Lab, UC San Diego}{June 2022 -- December 2022}
      {Undergraduate Research Assistant}{San Diego, CA}
      \resumeItemListStart
        \resumeItem{Performed hands-on analysis of ceramic tile properties and refractory wear to improve product quality.}
        \resumeItem{Optimized kiln firing profiles for industrial-grade ceramic tiles to enhance product durability and process efficiency.}
        \resumeItem{Utilized Scanning Electron Microscopy (SEM) and X-ray Diffraction (XRD) for material characterization, documenting findings in detailed technical reports for faculty review.}
      \resumeItemListEnd
    \resumeSubHeadingListEnd
\vspace{-16pt}

%-----------PROJECTS-----------%
\section{Projects}
  \resumeSubHeadingListStart
    \resumeSubheading
      {Ceramic Tile Quality Optimization}{Fall 2022}
      {Material Analysis, Kiln Process Optimization, Quality Control}{}
      \resumeItemListStart
        \resumeItem{Conducted comprehensive process optimization and quality analysis for commercial ceramic tiles.}
        \resumeItem{Performed hands-on analysis of tile properties and refractory wear to identify areas for improvement in a simulated production environment, boosting potential yield by 10\%.}
    \resumeItemListEnd
    
    \resumeSubheading
      {Kiln Performance and Refractory Enhancement}{Spring 2023}
      {Refractory Testing, Kiln Optimization, Material Selection}{}
      \resumeItemListStart
        \resumeItem{Applied expertise in selecting and testing refractory materials to optimize kiln performance and improve the quality of high-volume ceramic tile production.}
        \resumeItem{Optimized kiln firing profiles for industrial-grade ceramic tiles and castable refractories, enhancing product durability and process efficiency.}
      \resumeItemListEnd
  \resumeSubHeadingListEnd
\vspace{-16pt}

\section{Awards \& Achievements}
\begin{itemize}[leftmargin=0.35in, itemsep=0pt, label={\tiny$\bullet$}]
    \item \small\textbf{Senior Design Project Award} \hfill \small\textbf{May 2023} \\
    Awarded for developing an innovative method for testing the thermal shock resistance of industrial refractory bricks, resulting in a 15\% improvement in material selection accuracy.
\end{itemize}
   
\vspace{-16pt}

%-----------PROGRAMMING SKILLS-----------
\section{Technical Skills}
 \begin{itemize}[leftmargin=0.35in, itemsep=0pt, label={\tiny$\bullet$}]
     \item \textbf{Analysis}{: Ceramic Material Analysis, Refractory Selection \& Testing, Quality Control, Technical Reporting}
     \item \textbf{Characterization}{: Scanning Electron Microscopy (SEM), X-ray Diffraction (XRD), Thermal Analysis}
     \item \textbf{Processes}{: Kiln Process Optimization, Ceramic Processing \& Design, Furnace Technology}
     \item \textbf{Software}{: Process Simulation Software (e.g., COMSOL), MATLAB, Python, MS Office Suite}
 \end{itemize}
 \vspace{-16pt}

%-----------INVOLVEMENT---------------
\section{Extracurricular Activities}
    \resumeSubHeadingListStart
        \resumeSubheading{American Ceramic Society (ACerS) Student Chapter}{September 2020 -- January 2023}{Active Member}{UC San Diego}
            \resumeItemListStart
                \resumeItem{Actively participated in workshops and site visits to local manufacturing facilities.}
                \resumeItem{Gained practical insight into industrial ceramic processing, quality control, and large-scale applications.}
            \resumeItemListEnd
    \resumeSubHeadingListEnd

\end{document}